\newpage
\section{Bitwise Operation (20 pts)} 

Consider the following program for bitwise operation.

\resetlinenumber[1]
\linenumbers
\begin{tt}
	\lstinputlisting{\basepath/q3/q3.c}
\end{tt}
\nolinenumbers

The {\tt findOdd} function takes an integer array as input. In the input array, there is one element that occurs odd times, while all other elements occur even times. The {\tt findOdd} function returns the only element that occurs odd times.
\\
~\\
The {\tt checkEven} function takes an integer as input and checks if the integer is an even number.
\\
~\\
Hint: the output of the given code is

\begin{tt}
<--- ANSWER C: [ANSWER C] --->\\
121 is an odd number.
\end{tt}
\\
~\\
{\bf ANSWER A: (8 pts)} Fill in the code in line 11. 

(Hint 1: Think about a case where every number appears in the array exactly two times, except for the number you want, which appears exactly once.)

(Hint 2: Suppose the number you're looking for has a 1 as its seventh bit. What can you say about how many times the seventh bit will be 1 across all the numbers in the array? Suppose the number you're looking for has a 0 as its sixth bit. What can you say about how many times the sixth bit will be 1 across all the numbers in the array?)
\\
~\\
{\bf ANSWER B: (8 pts)} Fill in the code in line 19. For full credit, your answer should only use bitwise operations.
\\
~\\
{\bf ANSWER C: (4 pts)} What does line 30 print? Assume the code works. Your answer should only contain a hexadecimal number.

\ifexam


\else


{\bf Answer:}

Answer A: value \^{} arr[i]

Answer B: \& 1

Answer C: 79

\fi
