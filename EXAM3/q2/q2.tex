\newpage
\section{Binary Search Tree (25 pts)}

Consider the following definition in \texttt{tree.h}:

\resetlinenumber[1]
\linenumbers
\begin{tt}
	\lstinputlisting{\basepath/q2/tree.h}
\end{tt}
\nolinenumbers

and the following functions in {\tt tree.c} for binary search tree:

\resetlinenumber[1]
\linenumbers
\begin{tt}
	\lstinputlisting{\basepath/q2/q2.c}
\end{tt}
\nolinenumbers

There is a bug in {\tt tree\_insert} function, such that it cannot handle duplicate values properly. Do NOT fix the bug.

We use the given {\tt tree\_insert} function to insert nodes 5, 3, 5, 2, 4, 6, 3, 3, 5, 2 (in this order) to an empty binary search tree. Answer questions A and B.
\\
~\\
{\bf ANSWER A: (5 pts)} What is the \underline{last} number in the pre-order traversal? Your answer should only contain an integer.
\\
~\\
{\bf ANSWER B: (5 pts)} What is the \underline{fourth} number in the post-order traversal? Your answer should only contain an integer.
\\
~\\
After the above insert operations, we then use the given {\tt tree\_delete} function to delete nodes 5, 3, 1 (in this order). Answer questions C and D.
\\
~\\
{\bf ANSWER C: (7 pts)} What is the \underline{first} number in the pre-order traversal? Your answer should only contain an integer.
\\
~\\
{\bf ANSWER D: (8 pts)} What is the \underline{fourth} number in the post-order traversal? Your answer should only contain an integer.

\ifexam


\else


{\bf Answer:}

Answer A: 5

Answer B: 3

Answer C: 4

Answer D: 2


\fi
