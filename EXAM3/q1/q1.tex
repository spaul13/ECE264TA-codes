\newpage
\section{Binary Tree (30 pts)} 

Consider a binary tree whose post-order description is

H  D  E  B  I  J  F  G  C  A

The in-order description is

D  H  B  E  A  I  F  J  C  G

Please write down the pre-order descripition of the binary tree
and answer these questions.
\\
~\\
{\bf ANSWER A: (4 pts)} What is the root of the tree?
\\
~\\
{\bf ANSWER B: (6 pts)} What is the right child of the root of the tree?
Your answer should be a letter between A and H.
\\
~\\
{\bf ANSWER C: (6 pts)} What is the left child of the root of the tree?
Your answer should be a letter between A and H.
\\
~\\
{\bf ANSWER D: (7 pts)} If a tree has the same in-order description and
post-order description, how large can this tree be (measured by the
number of nodes)?  Please choose one correct answer. Your answer
should be a number 1, or 2, or 3 ... or 9.

\begin{enumerate}
\item 0 node
\item 1 node
\item 2 nodes
\item 3 nodes
\item 4 nodes
\item 7 nodes
\item 8 nodes
\item There is no limit
\item None of the above
\end{enumerate}

{\bf ANSWER E: (7 pts)} If a tree has the same pre-order description and
post-order description, how large can this tree be (measured by the
number of nodes)?  Please choose one correct answer. Your answer
should be a number 1, or 2, or 3 ... or 9.

\begin{enumerate}
\item 0 node
\item 1 node
\item 2 nodes
\item 3 nodes
\item 4 nodes
\item 7 nodes
\item 8 nodes
\item There is no limit
\item None of the above
\end{enumerate}

For your reference, the methods for generating the three descriptions
of a binary tree are shown below.  {\tt Node} is the data type for a
tree node.

\begin{verbatim}
void preorder(Node * tn)
{
    if (tn == NULL) { return; }
    print (tn -> data);
    preorder(tn -> left);
    preorder(tn -> right);
}

void inorder(Node * tn)
{
    if (tn == NULL) { return; }
    inorder(tn -> left);
    print (tn -> data);
    inorder(tn -> right);
}

void postorder(Node * tn)
{
    if (tn == NULL) { return; }
    postorder(tn -> left);
    postorder(tn -> right);
    print (tn -> data);
}
\end{verbatim}


\ifexam


\else


{\bf Answer:}

ANSWER A: A

Answer B: C

Answer C: B

Answer D: 8 (no limit)

Answer E: 2 (1 node)


\fi
