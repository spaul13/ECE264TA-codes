\newpage
\section{Recursive Function}

Consider the following program.

\resetlinenumber[1]
\linenumbers
\begin{tt}
  \lstinputlisting{\basepath/q4/q4.c}
\end{tt}
\nolinenumbers

Suppose the executable program is named {\tt recprog}.

Hint: If we run the program with
\begin{verbatim}
	./recprog 1
\end{verbatim}
the output of the program will be
\begin{verbatim}
	Function call
	func(1) = 1
\end{verbatim}

{\bf ANSWER A}: Run the program with  
\begin{verbatim}
	./recprog 8
\end{verbatim}
What is the value of \texttt{func(8)}?

Your answer should only contain an integer, that is 
\texttt{
	func(8) = [ANSWER A]
}

\ifexam

\else

{\bf Answer: 21}

\fi


{\bf ANSWER B}: If we run the program with  
\begin{verbatim}
	./recprog 5
\end{verbatim}
How many lines of {\tt\bf Function call} are there in the output? For example, there is one line of {\tt\bf Function call} in the output if we run the program with \texttt{./recprog 1}. Your answer should only contain an integer.

\ifexam

\else

{\bf Answer: 15}

\fi


{\bf ANSWER C}: In general, denote \texttt{g(n)} as the number of lines of {\tt\bf Function call} in the output by running the program with integer \texttt{n}. What is the relation between \texttt{g(n)} and \texttt{g(n-1)} and \texttt{g(n-2)}? Your answer should be the number of the statement below.

\begin{enumerate}
\item \texttt{g(n) = g(n-1) + g(n-2)}
\item \texttt{g(n) = g(n-1) + g(n-2) + 1}
\item \texttt{g(n) = g(n-1) + 1}
\item \texttt{g(n) = g(n-1) + 2}
\end{enumerate}

\ifexam

\else

{\bf Answer: 2}

\fi